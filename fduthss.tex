\documentclass[UTF8, colorlinks, oneside]{fduthss}
\usepackage{algorithm}
\usepackage{algpseudocode}
\usepackage{amsmath}
\usepackage{amssymb}
\usepackage[
	backend = biber, style=caspervector, utf8, citestyle = numeric,
	hyperref=true, backref=true, sorting=none,
	giveninits = true, sortgiveninits = true
]{biblatex}
\usepackage{bm}% bold math
\usepackage{caption}
\usepackage{colortbl}
\usepackage{enumerate}
\usepackage{epsfig}
\usepackage{fancyvrb}
\usepackage{framed}
\usepackage{graphics}
\usepackage{graphicx}
\usepackage{hyperref}
\usepackage[utf8]{inputenc}
\usepackage{lmodern}
\usepackage{multirow}
\usepackage{paracol}
\usepackage{rotating} 
\usepackage{subcaption}
\usepackage{tabulary}
\usepackage[titletoc]{appendix}
\usepackage{url}

%\usepackage[monochrome]{color}
% 使被强调的内容为红色。hyperref 宏包在设置了 colorlinks 时会引入 xcolor 宏包,
% 后者定义了 \textcolor 命令,故不用单独引入宏包。
\newcommand{\myemph}[1]{\emph{\textcolor{red}{#1}}}

% todo 的标记
\newcommand{\todo}[1]{\emph{\textcolor{red}{TODO: #1}}}
\renewcommand{\baselinestretch}{1.5}

% fduthss 文档模版的版本。
\newcommand{\docversion}{v1.4 rc4}
% 设定文档的基本信息。
\fduthssinfo{
	cthesisname = {本科毕业论文}, ethesisname = {Undergraduate Thesis},
	ctitle = {复旦大学本科毕业论文模板},
	etitle = {Undergraduate Thesis Template of Fudan University},
	cauthor = {},
	eauthor = {void rank},
	studentid = {123456789},
	date = {2018年5月},
	school = {计算机科学技术学院},
	cmajor = {计算机科学与技术},
	direction = {模板},
	cmentor = {professor}, ementor = {professor},
  ckeywords = { 模板, \LaTeX},
  ekeywords = { TEMPLATE, \LaTeX}
}
\addbibresource{egbib.bib}
\begin{document}
	\frontmatter

	\maketitle
	% vim:ts=4:sw=4
%
% Copyright (c) 2008-2009 solvethis
% Copyright (c) 2010-2013 Casper Ti. Vector
% Public domain.

\begin{cabstract}

此处可以写你的中文综述。

\end{cabstract}

\begin{eabstract}

English Abstract Placeholder.

\end{eabstract}


	
	\tableofcontents

	\mainmatter
	
	\chapter{引言}\label{chap1}

本科毕业论文一般包括如下章节

\begin{enumerate}

\item 中/英文综述
\item 引言
\item 相关工作
\item 自己的工作1
\item 自己的工作2
\item ..
\item 总结

\end{enumerate}


	\chapter{相关工作}\label{chap2}

本模板改自北京大学本科毕业论文模板pkuthss<https://gitlab.com/CasperVector/pkuthss>。

本模板修改了logo,适配了复旦大学本科毕业论文的排版(如1.5倍行距等)。

本模板修改自复旦计院民间流传的版本,感谢这些学长学姐。

版权如有问题,请指出,侵改删。


	\chapter{如何插入图片、表格、公式}\label{chap3}

请打开chap/chap3.tex(使用文本编辑器,IDE等)进行修改

\section{插入图片}

可以修改的部分
\begin{enumerate}
\item 图片路径: img/fduword\_black.png
\item 图片大小: width=0.8
\item 图注: caption
\item 图片引用锚: label
\end{enumerate}

\begin{figure}[!thbp]
    \begin{center}
    \includegraphics[width=0.8\textwidth]{img/fduword_black.png}
    \end{center}
    \caption{logo}
    \label{fig:fdulogo}
\end{figure}


\section{插入表格}

如果调不到自己想要的格式请直接google或者wiki


\begin{table}[h!]
  \begin{center}
    \caption{表注}
    \label{tab:example_tab}
    \begin{tabular}{l|r} % <-- Alignments: 1st column left, 2nd middle and 3rd right, with vertical lines in between
      type & c \\
      \hline
      a & 0.06 \\
      b & 0.03 \\
    \end{tabular}
  \end{center}
\end{table}

\section{公式}

如果调不到自己想要的格式请直接google或者wiki

\begin{equation}
\label{equa:example_equa}
\frac{\partial{\varepsilon}}{\partial{\bm{x}_L}} =
\frac{\partial \varepsilon}{\partial \bm{x_L}}\frac{\partial x_L}{\partial x_l} =
\frac{\partial \varepsilon}{\partial \bm{x_L}}(1 + \frac{\partial }{\partial \bm{x}_l}\sum_{i=l}^{L-1} \mathcal{F}(\bm{x}_i, \mathcal{W}_i))
\end{equation}

	\chapter{添加引用文献}\label{chap4}

\section{去哪找?}

本模板使用bibtex添加引用文献。请到数据库找到bibtex格式(示例在egbib.bib,使用文本编辑器打开)

\section{如何添加}
把bibtex复制粘贴到egbib.bib中即可,引用锚是第一个关键字,如本示例的fastmask为示例的引用锚

	\chapter{如何在文中添加图片、图表、公式、文献的链接}\label{chap5}


\section{图片、图表、公式链接}

图片链接\ref{fig:fdulogo}

表格链接\ref{tab:example_tab}

公式链接\ref{equa:example_equa}

\section{文献链接}

文献链接\cite{fastmask}

  
  \appendix
  \emergencystretch=1em
	\printbibliography[heading = bibintoc]
	\backmatter
	
	% 致谢。
	\chapter{致谢}

请在此处开始你们的表演。
\end{document}

